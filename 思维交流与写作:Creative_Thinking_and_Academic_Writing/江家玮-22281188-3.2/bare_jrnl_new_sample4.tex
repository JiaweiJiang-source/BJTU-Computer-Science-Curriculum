% !TEX program = xelatex
\documentclass[lettersize,journal]{IEEEtran}
\usepackage{amsmath,amsfonts}
\usepackage{algorithmic}
\usepackage{algorithm}
\usepackage{array}
\usepackage{amsmath,amsfonts,amssymb,mathtools,bm}
\usepackage{stfloats}
\usepackage{url}
\usepackage{verbatim}
\usepackage{array,multirow}
\usepackage{textcomp}
\usepackage{threeparttable}
\usepackage{stfloats}
\usepackage{url}
\usepackage{verbatim}
\usepackage{graphicx}
\usepackage{cite}
\usepackage[caption=false,font=normalsize,labelfont=sf,textfont=sf]{subfig}
\usepackage{makecell}
\usepackage{soul}
\usepackage{colortbl,xcolor}
\usepackage{colortbl}
\usepackage[justification=centering]{caption}

% 支持中文
\usepackage{xeCJK}
\setCJKmainfont{Songti SC} % 宋体 for Chinese text
\setmainfont{Times New Roman} % IEEE uses Times New Roman for English

\newcommand{\bluecite}[1]{{\color{blue}\cite{#1}}}
\sethlcolor{yellow}

\bibliographystyle{ieeetr}

\hyphenation{op-tical net-works semi-conduc-tor IEEE-Xplore}

\begin{document}

\title{A Sample Article Using IEEEtran.cls\\ for IEEE Journals and Transactions}

\author{IEEE Publication Technology,~\IEEEmembership{Staff,~IEEE,}
\thanks{This paper was produced by the IEEE Publication Technology Group. They are in Piscataway, NJ.}
\thanks{Manuscript received April 19, 2021; revised August 16, 2021.}}

\markboth{Journal of \LaTeX\ Class Files,~Vol.~14, No.~8, August~2021}%
{Shell \MakeLowercase{\textit{et al.}}: A Sample Article Using IEEEtran.cls for IEEE Journals}

\IEEEpubid{0000--0000/00\$00.00~\copyright~2021 IEEE}

\maketitle

\begin{abstract}
This document describes the most common article elements and how to use the IEEEtran class with \LaTeX \ to produce files that are suitable for submission to the IEEE. IEEEtran can produce conference, journal, and technical note (correspondence) papers with a suitable choice of class options.
\end{abstract}

\begin{IEEEkeywords}
Article submission, IEEE, IEEEtran, journal, \LaTeX, paper, template, typesetting.
\end{IEEEkeywords}

\section{Introduction}
This file is intended to serve as a ``sample article file'' for IEEE journal papers.

{\footnotesize
\begin{align}
\alpha_{m,n,j}^{(l)} &= \frac{\exp\left(f_2\left(f_1\left((\bm{w}_{m,\text{dir}}^{(l)} + \bm{w}_{m,\text{ner}}^{(l)}\bm{h}_j^{(l)})^T \bm{a}_m^{(l)}\right)\right)\right)}{\sum_{i \in \mathbb{N}(n)} \exp\left(f_2\left(f_1\left((\bm{w}_{m,\text{dir}}^{(l)} + \bm{w}_{m,\text{ner}}^{(l)}\bm{h}_i^{(l)})^T \bm{a}_m^{(l)}\right)\right)\right)} \label{eq:alpha} \\
\beta_{m,n}^{(l)} &= f_3\left(\alpha_{m,n,j}^{(l)}, \bm{w}_{m,\text{ner}}^{(l)}\bm{h}_j^{(l)} \mid j \in \mathbb{N}(n)\right) \label{eq:beta} \\
\phi(\bm{w}_n^{(C)}) &= \begin{cases}
\sqrt{P_n} \bm{w}_n^{(C)}, & \|\bm{w}_n^{(C)}\|_2 \le 1, \\
\frac{\sqrt{P_n} \bm{w}_n^{(C)}}{\|\bm{w}_n^{(C)}\|_2}, & \|\bm{w}_n^{(C)}\|_2 > 1.
\end{cases} \label{eq:phi}
\end{align}
}
IEEEtran.cls version 1.8b and later. The most common elements are covered in the simplified and updated instructions in ``New\_IEEEtran\_how-to.pdf''. For less common elements you can refer back to the original ``IEEEtran\_HOWTO.pdf''. It is assumed that the reader has a basic working knowledge of \LaTeX. Those who are new to \LaTeX \ are encouraged to read Tobias Oetiker's ``The Not So Short Introduction to \LaTeX ,'' available at: \url{http://tug.ctan.org/info/lshort/english/lshort.pdf} which provides an overview of working with \LaTeX.
\begin{align}
s_{m,i}^{(l)} &= (\bm{w}_{m,\text{dir}}^{(l)} + \bm{w}_{m,\text{ner}}^{(l)}\bm{h}_i^{(l)})^T \bm{a}_m^{(l)}. \label{eq:s_def} \\
\alpha_{m,n,j}^{(l)} &= \frac{\exp\left(f_2\left(f_1\left(s_{m,j}^{(l)}\right)\right)\right)}{\sum_{i \in \mathbb{N}(n)} \exp\left(f_2\left(f_1\left(s_{m,i}^{(l)}\right)\right)\right)} \label{eq:alpha_new} \\
\beta_{m,n}^{(l)} &= f_3\left(\alpha_{m,n,j}^{(l)}, \bm{w}_{m,\text{ner}}^{(l)}\bm{h}_j^{(l)} \mid j \in \mathbb{N}(n)\right) \label{eq:beta_repeat} \\
\phi(\bm{w}_n^{(C)}) &= \begin{cases}
\sqrt{P_n} \bm{w}_n^{(C)}, & \|\bm{w}_n^{(C)}\|_2 \le 1, \\
\frac{\sqrt{P_n} \bm{w}_n^{(C)}}{\|\bm{w}_n^{(C)}\|_2}, & \|\bm{w}_n^{(C)}\|_2 > 1.
\end{cases} \label{eq:phi_repeat}
\end{align}

\begin{figure}[t]
\centering
\includegraphics[width=0.48\textwidth]{attention.pdf}
\caption{The processes of each CGAL for one node.}
\label{fig:attention}
\end{figure}

\begin{table}[H]
\begin{threeparttable}
\centering
\begin{tabular}{l| c|c| c |c |c}
\hline
\multirowcell{2}{DL Model} & \multicolumn{4}{c|}{Complexity and Efficiency Metrics} & \multirowcell{2}{BER$^\dagger$} \\
\cline{2-5}
& \begin{tabular}{@{}c@{}}FLOPs \\ {[giga]}\end{tabular}
& \begin{tabular}{@{}c@{}}Params \\ {[million]}\end{tabular}
& \begin{tabular}{@{}c@{}}IT \\ {[ms]}\end{tabular}
& \begin{tabular}{@{}c@{}}MU \\ {[MB]}\end{tabular}
& \\
\hline\hline
MaxMB & \cellcolor{pink!40}0.73 & \cellcolor{blue!20}0.45 & 20.45 & \cellcolor{blue!20}167.62 & \cellcolor{blue!20}0.1179 \\
\hline
DNN & 1.20 & \cellcolor{blue!20}0.60 & 1.78 & \cellcolor{blue!20}160.13 & 0.3836 \\
\hline
DeepRx & 1.37 & 0.66 & \cellcolor{pink!40}10.15 & 320.00 & \cellcolor{pink!40}0.1194 \\
\hline
SwitchNet & \cellcolor{blue!20}0.02 & 8.71 & \cellcolor{blue!20}0.12 & 321.84 & 0.1849 \\
\hline
Neural Receiver & 2.40 & 3.25 & 4.19 & 311.22 & 0.1219 \\
\hline
SCBiGNet & 0.87 & 1.04 & 4.23 & 878.59 & 0.1308 \\
\hline
\end{tabular}
\begin{tablenotes}
\item[$\dagger$] BER refers to the average BER achieved by each model across all SNR ranges.
\item IT/MU: Inference time/Memory usage.
\end{tablenotes}
\end{threeparttable}
\end{table}

\begin{algorithm}[H]
\caption{Cuckoo Search via Levy Flights}
\label{alg:cuckoo}
\begin{algorithmic}[1]
\STATE 目标函数$f(\bm{x})$, $\bm{x}=(x_1, \dots, x_d)^T$
\STATE 产生$n$个寄主的初始种群$\bm{x}_i$
\WHILE{$t < \text{MaxGeneration}$ or (stop criterion)}
\STATE 随机取一个布谷鸟
\STATE 通过Levy飞行产生一个解
\STATE 评估解的质量或目标函数值$f_i$
\STATE 从$n$个巢中随机选择一个 (假设为$j$)
\IF{$f_i < f_j$}
\STATE 将$j$用解$i$代替
\ENDIF
\STATE 一部分 ($p_a$) 糟糕的巢被抛弃
\STATE 新巢/解由式(1)产生
\STATE 保存最佳的解
\STATE 排列解找出当前最佳
\STATE 更新$t \leftarrow t+1$
\ENDWHILE
\STATE 后处理与可视化
\end{algorithmic}
\end{algorithm}

\vfill
\end{document}


% % !TEX program = xelatex
% \documentclass[lettersize,journal,utf8]{IEEEtran}
% \usepackage{amsmath,amsfonts}
% \usepackage{algorithmic}
% \usepackage{xeCJK}
% \usepackage{times}
% % \setCJKmainfont{PingFang SC} 
% \setCJKmainfont{Songti SC}   % 宋体
% % \setCJKmainfont{Heiti SC}
% \usepackage{algorithm}
% \usepackage{algorithmic}
% \usepackage{array}
% \usepackage[caption=false,font=normalsize,labelfont=sf,textfont=sf]{subfig}
% \usepackage{multirow}
% \usepackage{textcomp}
% \usepackage{threeparttable}
% \usepackage{stfloats}
% \usepackage{url}
% \usepackage{verbatim}
% \usepackage{graphicx}
% \usepackage{cite}
% \usepackage{amsmath, amssymb}
% \usepackage{bm}
% \usepackage{mathtools} 
% \usepackage{xcolor}
% \usepackage{colortbl}
% \usepackage{colortbl} 
% \usepackage{xcolor}
% \usepackage{makecell}
% \usepackage{bm}  
% \usepackage{soul}
% \usepackage{xcolor}
% \sethlcolor{yellow}
% \usepackage[justification=centering]{caption}
% \usepackage{xcolor}
% \newcommand{\bluecite}[1]{{\color{blue}\cite{#1}}}
% \usepackage{multirow}
% \bibliographystyle{ieeetr}


% \hyphenation{op-tical net-works semi-conduc-tor IEEE-Xplore}
% % updated with editorial comments 8/9/2021

% \begin{document}

% \title{A Sample Article Using IEEEtran.cls\\ for IEEE Journals and Transactions}

% \author{IEEE Publication Technology,~\IEEEmembership{Staff,~IEEE,}
%         % <-this % stops a space
% \thanks{This paper was produced by the IEEE Publication Technology Group. They are in Piscataway, NJ.}% <-this % stops a space
% \thanks{Manuscript received April 19, 2021; revised August 16, 2021.}}

% % The paper headers
% \markboth{Journal of \LaTeX\ Class Files,~Vol.~14, No.~8, August~2021}%
% {Shell \MakeLowercase{\textit{et al.}}: A Sample Article Using IEEEtran.cls for IEEE Journals}

% \IEEEpubid{0000--0000/00\$00.00~\copyright~2021 IEEE}
% % Remember, if you use this you must call \IEEEpubidadjcol in the second
% % column for its text to clear the IEEEpubid mark.

% \maketitle

% \begin{abstract}
% This document describes the most common article elements and how to use the IEEEtran class with \LaTeX \ to produce files that are suitable for submission to the IEEE.  IEEEtran can produce conference, journal, and technical note (correspondence) papers with a suitable choice of class options. 
% \end{abstract}

% \begin{IEEEkeywords}
% Article submission, IEEE, IEEEtran, journal, \LaTeX, paper, template, typesetting.
% \end{IEEEkeywords}

% \section{Introduction}
% \IEEEPARstart{T}{his} file is intended to serve as a ``sample article file''
% for IEEE journal papers.

% {\footnotesize
% \begin{align}
%     \alpha_{m,n,j}^{(l)} &= \frac{\exp\left(f_2\left(f_1\left((\bm{w}_{m,\text{dir}}^{(l)} + \bm{w}_{m,\text{ner}}^{(l)}\bm{h}_j^{(l)})^T \bm{a}_m^{(l)}\right)\right)\right)}{\sum_{i \in \mathbb{N}(n)} \exp\left(f_2\left(f_1\left((\bm{w}_{m,\text{dir}}^{(l)} + \bm{w}_{m,\text{ner}}^{(l)}\bm{h}_i^{(l)})^T \bm{a}_m^{(l)}\right)\right)\right)} \label{eq:alpha} \\
%     \beta_{m,n}^{(l)} &= f_3\left(\alpha_{m,n,j}^{(l)}, \bm{w}_{m,\text{ner}}^{(l)}\bm{h}_j^{(l)} \mid j \in \mathbb{N}(n)\right) \label{eq:beta} \\
%     \phi(\bm{w}_n^{(C)}) &= \begin{cases}
%         \sqrt{P_n} \bm{w}_n^{(C)}, & \|\bm{w}_n^{(C)}\|_2 \le 1, \\
%         \frac{\sqrt{P_n} \bm{w}_n^{(C)}}{\|\bm{w}_n^{(C)}\|_2}, & \|\bm{w}_n^{(C)}\|_2 > 1.
%     \end{cases} \label{eq:phi}
% \end{align}
% }

% \IEEEPARstart{T}{his} file is intended to serve as a ``sample article file''
% for IEEE journal papers produced under \LaTeX\ using
% IEEEtran.cls version 1.8b and later. 
% % 由于第一个公式太长了,因此我把它进行了分离
% \begin{align}
%   s_{m,i}^{(l)} &= (\bm{w}_{m,\text{dir}}^{(l)} + \bm{w}_{m,\text{ner}}^{(l)}\bm{h}_i^{(l)})^T \bm{a}_m^{(l)}. \label{eq:s_def} \\ % 在等号前加&
%     \alpha_{m,n,j}^{(l)} &= \frac{\exp\left(f_2\left(f_1\left(s_{m,j}^{(l)}\right)\right)\right)}{\sum_{i \in \mathbb{N}(n)} \exp\left(f_2\left(f_1\left(s_{m,i}^{(l)}\right)\right)\right)} \label{eq:alpha_new} \\
%     \beta_{m,n}^{(l)} &= f_3\left(\alpha_{m,n,j}^{(l)}, \bm{w}_{m,\text{ner}}^{(l)}\bm{h}_j^{(l)} \mid j \in \mathbb{N}(n)\right) \label{eq:beta} \\
%     \phi(\bm{w}_n^{(C)}) &= \begin{cases}
%         \sqrt{P_n} \bm{w}_n^{(C)}, & \|\bm{w}_n^{(C)}\|_2 \le 1, \\
%         \frac{\sqrt{P_n} \bm{w}_n^{(C)}}{\|\bm{w}_n^{(C)}\|_2}, & \|\bm{w}_n^{(C)}\|_2 > 1.
%     \end{cases} \label{eq:phi}
% \end{align}

% \begin{figure}[!t]
%     \centering
%     \includegraphics[width=0.48\textwidth]{attention.pdf}
%     \caption{The processes of each CGAL for one node.}
%     \label{fig:attention}
% \end{figure}

% \begin{table}[!t]
%     \begin{threeparttable}
%         \centering
%         \begin{tabular}{l| c|c| c |c |c}
%             \hline
%             \multirowcell{2}{DL Model} & \multicolumn{4}{c|}{Complexity and Efficiency Metrics} & \multirowcell{2}{BER$^\dagger$} \\ 
%             \cline{2-5}
%             & \begin{tabular}{@{}c@{}}FLOPs \\ {[giga]}\end{tabular} 
%             & \begin{tabular}{@{}c@{}}Params \\ {[million]}\end{tabular} 
%             & \begin{tabular}{@{}c@{}}IT \\ {[ms]}\end{tabular} 
%             & \begin{tabular}{@{}c@{}}MU \\ {[MB]}\end{tabular}
%             & \\
%             \hline\hline
%             MaxMB           & \cellcolor{pink!40}0.73 & \cellcolor{blue!20}0.45 & 20.45 & \cellcolor{blue!20}167.62 & \cellcolor{blue!20}0.1179 \\
%             \hline
%             DNN             & 1.20 & \cellcolor{blue!20}0.60 & 1.78  & \cellcolor{blue!20}160.13 & 0.3836 \\
%             \hline
%             DeepRx          & 1.37 & 0.66 & \cellcolor{pink!40}10.15 & 320.00 & \cellcolor{pink!40}0.1194 \\
%             \hline
%             SwitchNet       & \cellcolor{blue!20}0.02 & 8.71 & \cellcolor{blue!20}0.12  & 321.84 & 0.1849 \\
%             \hline
%             Neural Receiver & 2.40 & 3.25 & 4.19  & 311.22 & 0.1219 \\
%             \hline
%             SCBiGNet        & 0.87 & 1.04 & 4.23  & 878.59 & 0.1308 \\
%             \hline
%         \end{tabular}
%         \begin{tablenotes}
%             \item[$\dagger$] BER refers to the average BER achieved by each model across all SNR ranges.
%             \item IT/MU: Inference time/Memory usage.
%         \end{tablenotes}
%     \end{threeparttable}
% \end{table}

% \begin{algorithm}[H]
%     \caption{Cuckoo Search via Levy Flights}
%     \label{alg:cuckoo}
%     \begin{algorithmic}[1] % [1] 表示显示行号
%         \STATE 目标函数$f(\bm{x})$, $\bm{x}=(x_1, \dots, x_d)^T$
%         \STATE 产生$n$个寄主的初始种群$\bm{x}_i$
%         \WHILE{$t < \text{MaxGeneration}$ or (stop criterion)}
%             \STATE 随机取一个布谷鸟
%             \STATE 通过Levy飞行产生一个解
%             \STATE 评估解的质量或目标函数值$f_i$
%             \STATE 从$n$个巢中随机选择一个 (假设为$j$)
%             \IF{$f_i < f_j$}
%                 \STATE 将$j$用解$i$代替
%             \ENDIF
%             \STATE 一部分 ($p_a$) 糟糕的巢被抛弃
%             \STATE 新巢/解由式(1)产生
%             \STATE 保存最佳的解 (或者, 高质量解的巢)
%             \STATE 排列解找出当前最佳
%             \STATE 更新$t \leftarrow t+1$
%         \ENDWHILE
%         \STATE 后处理与可视化
%     \end{algorithmic}
% \end{algorithm}



% \vfill

% \end{document}


