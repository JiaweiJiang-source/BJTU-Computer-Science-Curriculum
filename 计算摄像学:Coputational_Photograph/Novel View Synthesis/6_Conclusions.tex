% ======================================================================
% == Section: Conclusions ==
% ======================================================================

\section{Conclusions}
\label{sec:conclusions}

This paper has presented a comprehensive survey of the field of Novel View Synthesis (NVS), charting its evolution from early geometry-based methods through the advancements brought by deep learning, generative models, and most recently, diffusion models. My primary focus has been an in-depth analysis of Neural Radiance Fields (NeRF), a technique that has revolutionized the field by enabling unprecedented levels of photorealism and geometric consistency through implicit volumetric scene representation and differentiable rendering.

I examined the core principles underpinning NeRF and systematically reviewed the significant progress made by the research community in addressing its initial limitations. Key advancements were discussed in handling challenging real-world scenarios, including:
\begin{itemize}
    \itemsep0em % Reduce space between items
    \item Extending NeRF to represent large-scale, unbounded environments.
    \item Adapting NeRF to model dynamic elements and appearance variations within scenes.
    \item Improving reconstruction quality from sparse or few input views through priors and regularization.
    \item Drastically accelerating NeRF's training and inference speed via hybrid representations and optimized data structures, paving the way for real-time applications.
\end{itemize}
The quantitative comparisons, such as those presented in Section~\ref{sec:quantitative_comparison}, highlight the remarkable performance gains achieved by these NeRF variants across various benchmarks. % Reference the section with the table

Despite this rapid progress, several challenges and exciting avenues for future research remain. Achieving true real-time performance on diverse hardware, including mobile devices, while maintaining high fidelity remains an ongoing pursuit. Handling highly complex dynamic scenes with intricate motion, topology changes, and extreme lighting variations requires more robust solutions. Further improving generalization from extremely sparse or casually captured data is crucial for democratizing high-quality NVS. Moreover, enhancing the editability and controllability of NeRF representations—allowing intuitive manipulation of scene content, appearance, and lighting—is a key direction for practical content creation workflows. The deeper integration of powerful generative priors from diffusion models or GANs, advancements in theoretical understanding, and applications bridging NVS with robotics and simulation also represent promising frontiers.

In conclusion, NeRF has fundamentally reshaped the landscape of novel view synthesis and implicit 3D scene representation. The continued exploration of its capabilities and the ongoing efforts to overcome its limitations promise a vibrant future for this field, with profound implications for computer vision, computer graphics, virtual/augmented reality, and beyond.

% ======================================================================
% == End of Conclusions Section ==
% ======================================================================