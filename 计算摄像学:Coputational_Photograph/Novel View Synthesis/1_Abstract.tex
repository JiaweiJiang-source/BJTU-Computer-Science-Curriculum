\begin{abstract}
Novel View Synthesis (NVS), the generation of images from new viewpoints based on existing input views, is a cornerstone problem in computer vision and graphics with wide-ranging applications. This paper presents a comprehensive survey charting the evolution of NVS methodologies. This paper traces the historical progression through four distinct eras: early techniques rooted in geometric principles and Multi-View Stereo (MVS) (c. 2000-2010); the advent of regression-based deep learning approaches (c. 2010-2017); the rise of powerful generative models like GANs and Transformers (c. 2018-2022); and the current wave of diffusion model-based synthesis (c. 2021-present). Subsequently, this paper provide an in-depth analysis of Neural Radiance Fields (NeRF), a paradigm-shifting technique that has recently dominated the field. This paper discuss the specific application requirements driving NeRF research, delineate key challenges including model capacity allocation, noise handling, few-shot learning, and computational efficiency, and summarize corresponding solution strategies. Furthermore, I offer a structured taxonomy of contemporary NeRF-related works, categorizing them based on their focus on large-scale scenes, dynamic elements and complex lighting, few-shot reconstruction, and computational acceleration. This survey serves as a valuable resource for understanding the NVS landscape and navigating the rapidly expanding NeRF literature.\end{abstract}
% Keywords: Novel View Synthesis, Neural Radiance Fields, Generative Models, Diffusion Models, Computer Vision